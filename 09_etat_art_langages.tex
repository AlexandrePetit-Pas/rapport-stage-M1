\guillemotleft En informatique, une donnée est la représentation d'une information dans un programme\guillemotright ~\cite{Wiki}. Il est donc possible de traiter des données dans le langage de programmation de son choix. Il n'existe pas de réelles contraintes empêchant l'utilisation d'un langage de programmation.\\

En revanche certains langages et paradigmes de programmation sont plus propices aux traitements de données. C'est pourquoi, on parle généralement de Python, de R ou de SAS lorsque le sujet de la datavisualisation est abordé.
\begin{itemize}
\item Python~\cite{Python} est un langage de programmation interprété, orienté objet. Connu pour sa simplicité de lecture, le langage est grandement utilisé par la communauté scientifique tout particulièrement pour les calculs numériques. Par ailleurs, c'est un langage typé dynamiquement, c'est-à-dire que le type de l'objet n'est pas obligatoirement connu à l'avance. Parmi les types les plus représentatifs pour des données, on retrouve les entiers (\textit{int}), les tuples ou n-uplets \textit{tuple}, les tableaux dynamiques (\textit{list}), les tableaux associatifs (\textit{dict}) et les chaînes des caractères (\textit{string}). En plus de la programmation orientée objet, Python permet de programmer selon le paradigme de la programmation impérative. Étant un langage interprété, la traduction du code se fait sans compilateur. Cela permet de faciliter le portage sur n'importe quel système d'exploitation. C'est pourquoi Python est disponible sur n'importe quel système d'exploitation.
\item R~\cite{R} est un langage de programmation procédurale fréquemment utilisé dans le domaine des statistiques. C'est un langage typé dynamiquement qui permet de faciliter le traitement des volumes massifs de données. De plus, R est aussi un logiciel libre pour les statistiques et la science des données. R (le langage et le logiciel) est \textit{cross-platform}, c'est à dire qu'il est disponible sur n'importe quel système d'exploitation.
\item SAS~\cite{SAS} est un langage de programmation ainsi qu'une suite logicielle propriétaire développée par SAS Institute. La suite logicielle utilise le langage SAS pour de l'analyse de données, comme l'analyse prédictive ou l'informatique décisionnelle\footnote{\textit{Business Intelligence}}. SAS est disponible sur les systèmes d'exploitation Windows et UNIX et commence à être porté vers Mac (uniquement une version académique).
\end{itemize}