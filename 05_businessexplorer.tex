Suite à la création de \textit{finElink}, FRS récoltait des données de navigation et d'audience or celles-ci n'étaient pas exploitées. Il fallait donc développer une interface rassemblant les informations les plus pertinentes. Ces informations s'adresseront alors à deux "types" d'utilisateurs : en interne chez FRS mais aussi aux financeurs publics.\\

Pour les financeurs, la plateforme permettrait d'avoir par exemple, un visuel sur l'actualité de l'innovation, mais aussi sur la concordance entre l'argent déployé pour un dispositif et la demande des entreprises innovantes. Cela pourra les aider dans leurs prises de décisions, en accordant plus d'importance ou non à un secteur d'activité, au travail collaboratif ou encore à certaines zones géographiques.\\

Pour FRS, l'outil permettrait d'aller au plus près de l'information, pour anticiper les secteurs les plus porteurs mais aussi démarcher les entreprises des secteurs émergents. Cela leur apportera une aide à la décision sur la direction stratégique à prendre afin d'identifier les projets les plus porteurs à accompagner.\\

C'est dans ce contexte que l'outil \textit{Business Explore}r est né, ma mission consistant à mener à bien son développement et son intégration dans l'outil \textit{finElink}.\\

Comme expliqué précédemment\footnote{Chapitre 3 finElink}, les données stratégiques de FRS sont générées directement par la plateforme \textit{finElink} mais aussi indirectement par \textit{Google Analytics}. Il faut alors centraliser toutes ces données pour pouvoir traiter efficacement les informations. Vient alors le besoin de mettre en valeur des informations qui n'étaient pas explicitement présentes dans le système. Par ailleurs, afficher directement toutes les données n'a aucune plus-value, il faut donc convenir des informations pertinentes et les agréger pour pouvoir les exploiter.\\

C'est pourquoi la première problématique de la mission était de déterminer les informations importantes à récupérer, puis celles à créer à partir de l'existant.\\

Suite à cela, s'ajoute une problématique de mise en valeur, une difficulté récurrente en informatique. Communément appelée IHM\footnote{Interface Homme Machine}, il est primordial de maximiser le confort utilisateur en fluidifiant la navigation. Une navigation trop compliquée n'est jamais agréable pour l'utilisateur qui risque de ne pas se familiariser avec l'outil et de ne plus s'en servir.