Le projet innovant \textit{finElink} est né du constat de FRS Consulting, partagé par de nombreux acteurs de l'innovation~\cite{InnovFR}, que la multitude de dispositifs d'aide aux entreprises et à l'innovation et d'acteurs publics engagés rendent difficile la recherche et l'identification de financement par les entrepreneurs porteurs de projets innovants. 
Trois obstacles principaux ont été identifiés : le défaut d'information, le manque d'expertise, le besoin d'un réseau. Ces difficultés ne sont pas insurmontables dans l'absolu, mais elles demandent énormément de temps et de savoir-faire. C'est précisément pour cette raison que certaines entreprises innovantes font appel à des cabinets de conseil qui mutualisent les efforts nécessaires. Aujourd'hui, une solution technique leur permettant de surmonter ces obstacles et accessible à l'ensemble des acteurs de l'innovation est nécessaire pour renforcer la capacité d'innovation en Europe. Malgré une constante progression du nombre d'entreprises innovantes en Europe, 51\% des sociétés de plus de 10 salariés dans l'Union Européenne ne sont pas innovantes, et 10\% d'entre elles identifient l'accès au financement public comme l'obstacle principal à leur innovation~\cite{Eurostat}. La réponse conçue et développée par FRS Consulting est la plateforme \textit{finElink}.\\

\textit{finElink} est une plateforme applicative d'intermédiation du financement de l'innovation avec pour objectif socio-économique principal de renforcer la capacité d'innovation de la France et de l'Europe. À cette fin, cette plateforme web se propose d'améliorer l'accessibilité et la représentation des informations sur les dispositifs de financement des projets innovants, en évaluant les aides financières sur l'ensemble de leurs externalités (retour technologique, emplois, labellisation pour d'autres financement, etc.) et en recommandant les meilleures sources de financement à chaque projet innovant, et ce gratuitement pour l'utilisateur. Une fois les sources de financement identifiées, les services d'aide à la constitution de consortium et les outils de rédaction des dossiers de demande et de business plan seront proposés par \textit{finElink} afin d'accompagner les utilisateurs tout au long de leurs démarches de financement de l'innovation.\\

À terme, \textit{finElink} sera un nexus de l'innovation et permettra à l'ensemble des acteurs sur un même outil de rechercher et se faire conseiller des dispositifs de financement ; de s'informer sur la performance des aides financières ; d'identifier les partenaires idéaux pour maximiser les chances de réussite de leurs projets innovants et/ou créatifs ; et de collaborer à la rédaction de leur dossier de demande d'aide. Il introduira également un modèle innovant et disruptif dans le marché du conseil aux entreprises, permettant ainsi à FRS Consulting via ce projet stratégique de se distinguer de sa concurrence.\\

Le point fort de l'outil est son utilisation d'API\footnote{\textit{Application Programming Interface}, ou Interface de programmation} REST\footnote{Representational State Transfer, aussi appelé RESTful} mais aussi d'Open Data. Ces technologies permettent d'offrir un service de meilleure qualité en améliorant l'expérience utilisateur.\\

\textit{finElink} propose actuellement la possibilité de consulter directement les dispositifs sans passer par de la recommandation. Cela peut être utile à la fois pour les entreprises voulant s'informer des dispositifs actuellement disponibles mais aussi aux personnes internes à FRS qui peuvent alors consulter un annuaire.\\

On peut donc rapidement distinguer deux types de données de navigations sur \textit{finElink} : la navigation pour la recommandation et celle pour accéder à l'annuaire. À cela s'ajoute des données d'audience du site web générées par Google Analytics\footnote{Service gratuit d'analyse d'audience d'un site web ou d'une application}.\\