L'équipe du département R\&D s'inspire de la méthode agile pour organiser ses travaux sans toutefois respecter l'intégralité de ses principes. Certains principes sont adaptés au contexte de recherche et développement.\\

À la place des daily scrum, une réunion est organisée chaque début de semaine. L'objectif est de faire le point sur où en est chacun et ce qui prévu d'être fait dans la semaine.\\

Le développement de fonctionnalités se faisaient en général sur une à deux semaines, ce qui correspond à des sprints de courte durée. Je me suis impliqué à rendre quelque chose de livrable toutes les semaines de manière à avoir un travail incrémental, régulier et en constante progression.\\

Étant seul à travailler sur l'outil Business Explorer, organiser des poker planning n'avait pas lieu d'être. Il fallait que j'analyse la difficulté des tâches à faire dans la semaine et que j'en parle avec les membres de l'équipe lors de la réunion hebdomadaire pour trouver ensuite une solution.\\

Enfin, des cas d'utilisation n'ont pas été écrits pendant le stage car ils avaient déjà été préparés en amont. De plus, \textit{Business Explorer} est un outil principalement destiné aux internes de FRS Consulting, si des doutes subsistaient, je pouvais aller voir directement les utilisateurs finaux de l'outil.\\

Le projet \textit{finElink} utilise l'outil de gestion de version git et plus particulièrement, la plateforme Gitlab. Chaque membre de l'équipe à une branche pour développer ses fonctionnalités. Ensuite un merge est fait sur une branche de développement pour tester le tout sur le serveur de développement. Une fois que la fonctionnalité est validée, elle est envoyée sur la branche master qui correspond au serveur de production.\\