FRS Consulting est spécialisé en stratégie et financement de l'innovation, des mesures fiscales nationales aux subventions européennes. Le cabinet, fondé en 2011 par 3 experts reconnus de l'innovation, notamment en fiscalité de la recherche, accompagne les entreprises innovantes des biotechnologies et sciences de la vie, du digital, ainsi que les grands comptes industriels en France métropolitaine et d'Outre-Mer, en Belgique, Espagne et au Royaume-Uni. L'objectif de FRS est de booster la croissance des entreprises innovantes à fort potentiel, et ainsi de construire les leaders de demain.\\

FRS porte l'innovation dans son ADN. Tout son savoir-faire exprime cette conviction. Il est basé tout d'abord, sur une analyse rigoureuse des facteurs scientifiques, économiques et concurrentiels du projet d'innovation. Ce diagnostic permet ensuite d'établir une stratégie de financement public, à moyen et long terme, qui identifie les dispositifs de soutien à l'innovation, nationaux et européens, les plus adaptés au fur et à mesure du développement du projet. Cette compétence s'adresse aussi bien à des projets collaboratifs réunissant plusieurs partenaires qu'à des projets menés par une entreprise seule. Pour chaque projet accompagné, FRS assure un engagement complet et entier, avec une latitude à 360\degree.\\

L'humain - ses qualités, ses aspirations, la précision de sa réflexion - est au centre des préoccupations de FRS. Cet engagement assumé est symbolisé par son logo, l'homme en calligraphie chinoise. C'est pourquoi, l'équipe internationale de FRS est réunie autour de valeurs fortes - Confiance, Excellence, Audace. Elle est composée de docteurs et d'ingénieurs en sciences issus du monde de la recherche, notamment en biologie médicale, en informatique ou encore en processus industriels, d'experts en intelligence économique, ainsi que de spécialistes en levées de fonds publics, y compris d'experts-évaluateurs pour la Commission européenne.\\

FRS dispose également d'un écosystème riche favorisant les échanges entre les principaux acteurs européens de l'innovation : représentants institutionnels nationaux et européens, pôles de compétitivité (Cap Digital, Medicen, Finance Innovation, Alsace Biovalley, Systematic, etc.), laboratoires académiques (Institut de Cardio-métabolisme et Nutrition - ICAN, Centre d'Immunologie de Marseille- Luminy - CIML, INSERM, Laboratoire d' Informatique de Paris 6 - LIP6, Laboratoire EconomiX, etc.), think tanks, ainsi que des partenaires spécialisés. Fort de ce réseau, FRS est sollicité pour contribuer au débat sur l'innovation et la transition numérique, à Paris et Bruxelles. Ses analyses sont également régulièrement publiées dans la presse généraliste et spécialisée.
En parallèle, FRS développe son propre projet de recherche pour améliorer les connaissances dans l'évaluation des politiques publiques européennes de soutien à la recherche et à l'innovation.\\

FRS est convaincu du rôle économique et social que l'entrepreneuriat joue pour insuffler le dynamisme nécessaire à la croissance, développer de nouvelles pratiques et enrichir le tissu économique. Aussi, FRS s'engage depuis plusieurs années, auprès d'associations comme Réseau Entreprendre, ou de pépinières et hôtels d'entreprises afin de transmettre son savoir et son expérience auprès des jeunes créateurs de start-up. Ces échanges marqués de la forte implication de FRS démontrent régulièrement leur efficacité pour faire grandir des projets innovants à répercussions sociétales et environnementales.\\
L'entreprise est capable d'accompagner ses clients de l'identification des projets à la perception des fonds. Pour cela, elle fait appel à des consultants et rédacteurs scientifiques : deux profils complémentaires assurant la pluridisciplinarité de FRS.\\
Les rédacteurs possèdent une expertise dans leur domaine respectif et sont amenés à la mobiliser pour gérer les projets auxquels ils sont rattachés. Leur rôle est d'analyser, challenger, et renforcer le processus scientifique envisagé lors du projet et à terme de rédiger et de défendre le projet auprès des organismes financeurs.\\

Les consultants accompagnent les entreprises en collaboration avec les rédacteurs, néanmoins là ou le rédacteur n'intervient que sur des aspects techniques le consultant lui porte les jalons orientés business, il conduit le projet de l'élaboration jusqu'à sa toute fin et intervient surtout pour structurer la stratégie d'innovation afin quelle soit cohérente avec les objectifs de court, moyen et long-terme de l'entreprise.\\

C'est dans ce cadre que s'inscrit mon stage au sein du pôle R\&D de FRS. Ce pôle est constitué de :
\begin{itemize}
\item Gaétan Le Chat, responsable du pôle,
\item Kymble Christophe et Vincent Grollemund, 2 étudiants en thèse Cifre\footnote{Conventions Industrielles de Formation par la Recherche},
\item Cheikhouna Diouck et moi, stagiaires niveau Master 1.
\end{itemize}
L'équipe travaille en commun sur un outil, appelé \textit{finElink}, que je détaillerai ultérieurement. Cependant, chacun a une mission précise, ce qui fait que nous travaillons en autonomie sur des fonctionnalités différentes. Cela n'empêche en aucun cas l'entraide entre chaque membre de l'équipe, à la fois par la correction de \textit{bogues} mais aussi par le partage de connaissances ou d'opinions.