C'est avec un bilan positif que se conclut ce rapport de stage. La mission est toujours en cours à la date de remise de ce document et se terminera le 31 juillet.\\

Les notions apprises durant mon année de Master 1 m'ont permis d'avoir une approche différente des projets habituels. Il m'a fallu faire un réel travail de veille technologique afin de choisir l'outil le plus adapté à ce que je devais développer. De plus, mon apprentissage des \textit{design pattern} m'a permis de réaliser une conception plus générique qui est plus facile à adapter aux besoins par la suite.\\

Pour les langages et technologies utilisés, mon expérience personnelle en Python ne m'a posée aucun problème de développement. En revanche, il ne m'avait jamais été donné d'exploiter autant de structures de données ce qui est fort appréciable car en plus de gagner en connaissance sur ce point j'ai aussi pu progresser en programmation. Ayant beaucoup utilisé le Javascript pour des projets personnels ou scolaires, ce langage n'était pas nouveau pour moi, en revanche, c'était la première fois que je faisais un projet de cette envergure sans framework tels que React ou Angular. Il en va de même pour les SVG en Javascript, j'ai dû mobiliser une notion clé enseignée cette année : l'abstraction. Elle m'a permis de dépasser les blocages du début et d'assimiler rapidement les principes du SVG.\\

Certains concepts de la méthode agile ont pu par ailleurs être appliqués. C'est le cas par exemple des sprints. En revanche cette méthode n'a pas été exploitée entièrement car elle n'était pas adaptée au fonctionnement de l'équipe.\\