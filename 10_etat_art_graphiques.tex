Un graphique, ou représentation graphique de données, est une représentation visuelle de données chiffrées. Elle s'exprime de différentes manières suivant l'information que l'on souhaite afficher et le message à faire passer. Avant de parler plus en détails des graphiques, il est important de préciser qu'il en existe un trop grand nombre pour en faire une liste exhaustive. De plus, des graphiques peuvent être combinés pour en donner un nouveau.\\

C'est pourquoi les graphiques présentés ci-dessous ne représentent pas l'ensemble des possibilités mais introduisent les formes régulièrement utilisées dans le domaine de la datavisualisation.\\

On distingue quatre grands groupes de graphiques : 
\begin{itemize}
\item les graphiques de relation
\item les graphiques de composition
\item les graphiques de comparaison
\item les graphiques de répartition (ou distribution)
\end{itemize}
\bigskip
Les graphiques de relation permettent de mettre en valeur les liens entre plusieurs variables. Généralement, ces graphiques se limitent à un petit nombre de variables. Par exemple les nuages de points~\cite{scatterplot} (\textit{scatter plot}) représentent deux variables tandis que les graphiques à bulles~\cite{bubbleplot} (\textit{bubble plot}) permettent de représenter trois variables plus facilement.\\

Les graphiques de composition sont utilisés pour comparer une donnée dans un ensemble. Ils représentent généralement les valeurs relatives (pourcentage par rapport au total) mais peuvent également représenter des valeurs absolues. On distingue deux types de graphiques de composition : les statiques et ceux changeant dans le temps. Parmi les graphiques de composition statique, le plus utilisé est le camembert~\cite{piechart} (\textit{pie chart}) car c'est l'un des plus simples à représenter et à comprendre. Pour les graphiques de composition changeant dans le temps, le plus représentatif est le graphique d'aires empilées~\cite{stackedareaplot} (\textit{stacked area chart}) qui permet de représenter des valeurs relatives ou absolues.\\


Pour évaluer l'étendue des données entre elles, on utilise des graphiques de comparaison. Cela permet de trouver rapidement les valeurs les plus extrêmes et celles les plus basses. Ces graphiques permettent aussi de comparer l'évolution des valeurs dans le temps, généralement par augmentation, diminution ou stagnation d'une courbe. Le graphique de barres~\cite{barchart} (\textit{bar chart}) ainsi que le graphique de courbes~\cite{linechart} (\textit{line chart}) sont les plus utilisés parmi les graphiques de comparaison. Attention néanmoins à ne pas confondre un graphique de barres et un histogramme, deux graphiques qui utilisent des barres comme indicateur mais n'ayant pas le même objectif de représentation.\\

Les graphiques de distribution permettent de savoir comment une variable quantitative est répartie sur un axe défini. L'objectif est de mettre en valeur les caractéristiques particulières d'un jeu de données, comme son intervalle de valeur ou sa tendance générale. Le graphique de distribution le plus utilisé est l'histogramme~\cite{histogram}.