Le développement de l'application se fait sur un serveur de développement, utilisé uniquement par l'équipe de R\&D. Celui-ci contient une base de données similaire à celle du serveur de production. Cependant, il n'est pas pertinent de synchroniser les données car l'évolution de celles-ci ne sont pas suffisamment importantes pour justifier des coûts de migration (manuelle ou automatique). Avoir un serveur de développement distinct d'un serveur de production permet d'avoir un environnement le plus proche de celui de l'utilisateur. Cela réduit la probabilité d'apparition de nouvelles erreurs et du traditionnel "ça marchait sur ma machine pourtant". Un autre avantage du serveur de développement est de ne pas être limité par les capacités de son matériel. Le serveur a des capacités de calcul plus élevées qu'une machine personnelle, ce qui permet de minimiser les ralentissements des opérations sur des grandes structures de données.\\

Un outil est utilisé pour isoler l'environnement python du projet sur le serveur, il s'agit de \textit{virtualenv}. Celui-ci permet de garder les dépendances requises pour un projet dans un emplacement séparé qui ne concerne que celui-ci. Cela veut dire que deux projets peuvent utiliser une même bibliothèque python dans des versions différentes sans créer des conflits.\\

Pour le développement de l'application, j'ai choisi de travailler "en local". Cela signifie que l'éditeur de texte se trouvait sur ma machine et que le code est envoyé par protocole SFTP au serveur. Cela permet d'avoir l'éditeur de texte de mon choix et de ne pas passer par une connexion SSH pour écrire le code. Cependant, cette connexion SSH était utile pour gérer les fichiers statiques et lancer les tests de l'application.\\